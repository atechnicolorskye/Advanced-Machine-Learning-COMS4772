\documentclass[12pt]{article}

% fonts
\usepackage[scaled=0.92]{helvet}   % set Helvetica as the sans-serif font
\renewcommand{\rmdefault}{ptm}     % set Times as the default text font

% dmb: not mandatory, but i recommend you use mtpro for math fonts.
% there is a free version called mtprolite.

% \usepackage[amssymbols,subscriptcorrection,slantedGreek,nofontinfo]{mtpro2}

\usepackage[T1]{fontenc}
\usepackage{amsmath}
\usepackage{amsfonts}

% page numbers
\usepackage{fancyhdr}
\fancypagestyle{newstyle}{
\fancyhf{} % clear all header and footer fields
\fancyfoot[R]{\vspace{0.1in} \small \thepage}
\renewcommand{\headrulewidth}{0pt}
\renewcommand{\footrulewidth}{0pt}}
\pagestyle{newstyle}

% geometry of the page
\usepackage[top=1in, bottom=1in, left=1.2in, right=1.2in]{geometry}

% paragraph spacing
\setlength{\parindent}{0pt}
\setlength{\parskip}{2ex plus 0.4ex minus 0.2ex}

% useful packages
\usepackage{natbib}
\usepackage{epsfig}
\usepackage{url}
\usepackage{bm}


\begin{document}

\begin{center}
  \Large \textbf{Advanced Machine Learning} \\
  \Large \textbf{Project Update} \\
  \vspace{0.1in}
  \normalsize Si Kai Lee \\
  \normalsize sl3950 \\
\end{center}

In the process of writing the proposal, I skimmed through Kawaguchi 2016i\footnote{Kawaguchi, K., 2016. Deep Learning without Poor Local Minima. arXiv preprint arXiv:1605.07110.}. Since then, I had read the paper twice, once for the Columbia Advanced Machine Learning Seminar and once to start the project. However, I could not understand it as It was really dense and not very well written. Since then, I have spoken to Professor Hsu about switching to another framework to show that the local minimums of $\frac{1}{2}||Y - W_3 W_2 W_1 X||_F^2$ are actually global minimums. 

I was suggested to look at Ge et al.'s Matrix Completion has No Spurious Local Minimum\footnote{Ge, R., Lee, J.D. and Ma, T., 2016. Matrix Completion has No Spurious Local Minimum. arXiv preprint arXiv:1605.07272.} and I have been trying to work through the proofs of the paper ever since. I am pretty sure that Ge et al. have the signs wrong for equation 3.4 but it does not affect the main gist of the paper. Currently, I am struggling to understand the proof of claim 2f. I was also pointed to Sebastian Bubeck's blogpost on the geometry of linearised neural networks and from there I have started reading Hardt and Ma's Identity Matters in Deep Learning\footnote{Hardt, M. and Ma, T, 2016. Identity Matters in Deep Learning. arXiv preprint arXiv:1611.04231} to try to gain an understanding of their framework and Kawaguchi's framework. I am also quite positive that the fifth line of the proof for Hardt and Ma in Bubeck's post has a sign error too.

\end{document}
